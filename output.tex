% !TeX spellcheck = pt_BR
% !TEX encoding = UTF-8 Unicode

\documentclass[12pt,a4paper]{article}

% O comando abaixo suprime as respostas e soluções:
%\def\nosolution {}

\newcommand{\mysourceurl}{\url{https://www.overleaf.com/8867764558xrvqdfdvjqfk}}

\title{Banco de Exercícios de Estatística}
\author{REA-Stat -- Recursos Educacionais Abertos de Estatística \\ \url{https://rea-stat.github.io/}}

\usepackage[brazil]{babel}
\usepackage[utf8]{inputenc}
\usepackage[T1]{fontenc}
\usepackage{lmodern}

\usepackage{color} 
\usepackage{dsfont}
\usepackage{setspace}
\usepackage{xcolor}

\usepackage{amsmath}
\usepackage{amssymb}
\usepackage{amsthm}
\usepackage{chngcntr}
\usepackage{dsfont}
\usepackage{graphicx}
\usepackage{float}
\usepackage{mathrsfs}  
\usepackage{microtype}
\usepackage{verbatim}

% Tira o espaço depois da vírgula:
\usepackage{icomma}

% Muda as margens:
\usepackage{geometry}
\geometry{margin=25mm,centering,a4paper}

\usepackage[shortlabels]{enumitem}
\setlist{topsep=0em, partopsep=.5em, itemsep=0pt, noitemsep}
\setlist[enumerate,1]{label=\rm(\alph*)}
\setlist[enumerate,2]{label=(\roman*)}

\usepackage{hyperref}
\hypersetup{
hidelinks,
colorlinks,
urlcolor=[rgb]{0,0,.7},
linkcolor=[rgb]{.4,0,0},
citecolor=[rgb]{0,.3,0},
pdfstartview=FitH,
pdfpagemode=UseNone
}

\renewcommand{\baselinestretch}{1.1}
\setlength{\parindent}{0pt}
\setlength{\parskip}{.4em}

\renewcommand*{\url}[1]{\href{{#1}}{\sf\footnotesize\detokenize{#1}}}

\graphicspath{{figures/}}

\theoremstyle{definition}
\newtheorem{exercise}{Exercício}
\newtheorem*{solution}{Solução}
\newtheorem*{answer}{Resposta}

\ifdefined\nosolution \let\solution=\comment \let\endsolution=\endcomment \fi
\ifdefined\nosolution \let\answer=\comment \let\endanswer=\endcomment \fi

\usepackage[main,input,include]{embedall}
%Use \embedfile{xxx.pdf} to manually embed figures!


\begin{document}


\maketitle

\textbf{Fique à vontade para contribuir com seus exercícios!}

Para isso, visite
\url{https://www.overleaf.com/8867764558xrvqdfdvjqfk}

Ao incluir um exercício no \href{https://www.overleaf.com/8867764558xrvqdfdvjqfk}{Overleaf}, você concorda que o direito autoral será do Projeto REA-Stat e ele será publicado sob a licença \href{https://creativecommons.org/licenses/by-sa/4.0/deed.pt-br}{CC-BY-SA}.

O material a ser contribuído tem que ser escrito por você mesmo, não pode ser copiado.

Se tem um exercício do qual você gosta muito, mas que não é seu, você pode reescrevê-lo com suas próprias palavras, sem copiar. O direito autoral protege uma expressão criativa, não uma ideia: é permitido reutilizar a mesma ideia se você expressá-la de outra forma.

\tableofcontents

\clearpage
\section{Licença e créditos}

\noindent
\copyright\
2023--{\the\year}
REA-Stat
\\

\vspace{1em}

\hfil
\href{https://creativecommons.org/licenses/by-sa/4.0/deed.pt-br}{\includegraphics[height=4.2em]{by-sa.pdf}}
\embedfile{figures/by-sa.pdf}

\vspace{1em}

\noindent
Todos os direitos reservados.
Permitido o uso nos termos licença Creative Commons Atribuição-CompartilhaIgual 4.0 Internacional.

\vspace{1em}

\textbf{Reutilização deste material}
\\
\noindent
Você pode remixar, transformar, e criar a partir do material para qualquer fim, mesmo que comercial.
Nesse caso, tem de distribuir as suas contribuições sob a mesma licença que o original.
Você não pode aplicar termos jurídicos ou medidas de caráter tecnológico que restrinjam legalmente outros de fazerem algo que a licença permita.

\vspace{1em}

\textbf{Atribuição}
\\
\noindent
Este material é uma produção coletiva do projeto
\\
\emph{REA-Stat -- Recursos Educacionais Abertos de Estatística}
(\url{https://rea-stat.github.io/}).

\vspace{1em}

\textbf{Código-fonte}
\\
\noindent
O código-fonte deste material está disponível em:
\mysourceurl


\vspace{1em}

\textbf{Aviso legal}
\\
\noindent
As pessoas e instituições aqui mencionadas não endossam a qualidade deste material e as opiniões nele contido, nem explícita nem implicitamente.
Este material pode conter erros.

\vspace{1em}

\hfill
\today

\vfill
\vfill

\thispagestyle{empty}

\clearpage


\section{Como usar exercícios deste banco}

Para provas ou listas de exercícios, copiar o seguinte em cada exercício ou no cabeçalho do material todo:

\hfil
\href{https://creativecommons.org/licenses/by-sa/4.0/deed.pt-br}{\includegraphics[height=1em]{by-sa.pdf}}
\href{https://rea-stat.github.io/}{REA-Stat}

Importante: se for no cabeçalho do documento, você estará publicando toda a lista de exercícios sob a licença CC-BY-SA. Neste caso, a sua lista não deve conter exercícios de autor desconhecido ou copiados de livros.

\medskip

A primeira linha no preâmbulo do código-fonte deste documento contém um comando que faz as soluções e respostas desaparecerem: \verb|\def\nosolution {}|


\clearpage
\section{Probabilidade}

\begin{exercise}
Considere o experimento resultante do lançamento de dois dados onde se
observa o mínimo entre suas faces. Construa um modelo probabilístico associado.
\end{exercise}

\begin{exercise}
Se $P(A)=P(A|B)=\frac{1}{4}$ e $P(B|A)=\frac{1}{2}$:
\begin{enumerate}
\item $A$ e $B$ são independentes?
\item $A$ e $B$ são mutuamente exclusivos?
\item Calcule $P(A^c|B^c)$.
\end{enumerate}
\end{exercise}

\begin{exercise}
Em uma gaveta existem 2 maços de baralho fechados.
Um deles é um baralho comum de 52 cartas, $\{A,2,3,\dots,9,10,J,Q,K\} \times \{ \clubsuit,\heartsuit,\spadesuit,\diamondsuit \}$, e outro é um baralho de truco com 40 cartas (não possui as cartas de números `8', `9' e `10').

Um dos maços é retirado da gaveta ao acaso e depois uma carta é sorteada ao acaso do baralho retirado.

\begin{enumerate}[(a)]
\item
Calcule a probabilidade de a carta sorteada ser uma das três figuras reais ($J,Q,K$).
\item
Sabendo-se que foi sorteada uma figura real, calcule a probabilidade de o baralho retirado ter sido o baralho comum.
\item
Calcule a probabilidade de a carta sorteada ser de espadas $\spadesuit$.
\item
Sabendo-se que foi sorteada uma carta de espadas, calcule a probabilidade de o baralho retirado ter sido o baralho de truco.
\item
Sejam $A=$ ``Foi retirado o baralho comum'',
$B=$ ``Foi sorteada uma figura real'' e
$C=$ ``Foi sorteada uma carta de espadas''.
$A$ e $B$ são independentes?
$A$ e $C$ são independentes?
$A$, $B$ e $C$ são coletivamente independentes?
\item
Qual a probabilidade de se sortear uma carta de número `5' ?
\item
Sabendo-se que foi sorteado um número (i.e., não foi sorteado $A$, $J$, $Q$ nem $K$), qual a probabilidade de o baralho retirado ter sido o baralho de truco?
\end{enumerate}
\end{exercise}

\begin{exercise}
\end{exercise}

\clearpage
\section{Variáveis aleatórias}

\begin{exercise}
Seja $X$ o número de caras obtidas em 4 lançamentos de uma moeda honesta.
Determine a função de probabilidade de $X$.
Desenhe o gráfico da função de distribuição da variável aleatória $X$.
\end{exercise}

\begin{exercise}
Se $X\sim\exp(\lambda)$ e $Y=5X$, ache a distribuição acumulada de~$Y$.
Ache a função de distribuição condicional e a densidade condicional de $Y$ dado que~$X>3$.
\end{exercise}

\begin{exercise}
Sejam $X$ e $Y$ variáveis aleatórias discretas e independentes.
Mostre que
$$ p_{X+Y}(t) = \sum_s p_X(s) \cdot p_Y(t-s).$$
Sugestão: particione $\Omega$ segundo o valor de $X$.
\end{exercise}

\begin{exercise}
O número $X$ de uvas-passas encontradas em um panetone tem distribuição $\mathrm{Poisson}(\lambda)$.
O panetone, estando com a data de validade vencida há alguns meses,
pode ter uvas-passas estragadas.
Cada uva-passa pode estar estragada independente das demais, com probabilidade~$p$.
Encontre a distribuição do número de uvas-passas estragadas e calcule a probabilidade de não haver nenhuma estragada.
\end{exercise}

\begin{exercise}
\end{exercise}


\clearpage
\section{Esperança}

\begin{exercise}
Considere o seguinte jogo de azar.
Uma urna contém $18$ bolas, sendo $9$ azuis e $9$ brancas.
Retiram-se $3$ bolas da urna ao acaso.
As bolas retiradas são descartadas e o jogador marca $1$ ponto se pelo menos $2$ dessas $3$ bolas forem azuis.
Em seguida retiram-se outras $3$ bolas da urna ao acaso, as bolas retiradas são descartadas e o jogador marca $1$ ponto se pelo menos $2$ dessas $3$ bolas forem azuis.
Repete-se o procedimento até que a urna esteja vazia.
Ao final, o jogador recebe um prêmio $X$ igual ao total de pontos marcados.
Calcule $EX$.
\end{exercise}

\begin{exercise}
\end{exercise}


\clearpage
\section{Distribuição normal}

\begin{exercise}
A distribuição dos comprimentos dos elos da corrente de bicicleta 
é normal, com média $2$ cm e variância $0,01$ $cm^{2}$. Para que
uma corrente se ajuste à bicicleta, deve ter comprimento total entre $58$
e $61$ cm. Qual é a probabilidade de uma corrente com $30$ elos não
se ajustar à bicicleta?
\end{exercise}

\begin{exercise}
As durações de gravidez têm distribuição normal com
média de 268 dias e desvio-padrão de $15$ dias.

(a) Selecionada aleatoriamente uma mulher grávida, determine a
probabilidade de que a duração de sua gravidez seja inferior a $260$
dias.

(b) Se $25$ mulheres escolhidas aleatoriamente são submetidas a uma
dieta especial a partir do dia em que engravidam, determine a probabilidade
de os prazos de duração de suas gravidezes terem média inferior
a $260$ dias (admitindo-se que a dieta não produza efeito).

(c) Se as $25$ mulheres têm realmente média inferior a $260$ dias,
há razão de preocupação para os médicos de pré-natal?
Justifique adequadamente.
\end{exercise}

\begin{exercise}
O peso de uma determinada fruta é uma variável aleatória com
distribuição normal com média de $200$ gramas e desvio-padrão de
$50$ gramas. Determine a probabilidade de um lote contendo $100$
unidades dessa fruta pesar mais que $21$ kg.
\end{exercise}

\begin{exercise}
Um elevador pode suportar uma carga de $10$ pessoas ou um peso total de $1750 $
libras. Assumindo que apenas homens tomam o elevador e que seus pesos
são normalmente distribuídos com média $165$ libras e
desvio-padrão de $10$ libras, qual a probabilidade de que o peso limite
seja excedido para um grupo de $10$ homens escolhidos aleatoriamente?
\end{exercise}

\begin{exercise}
Um par de dados honestos é lançado $180$ vezes por hora.
\begin{enumerate}
\item
Qual a probabilidade aproximada de que $25$ ou mais lançamentos tenham tido soma $7$ na primeira hora?
\item
Qual a probabilidade aproximada de que entre $700$ e $750$ lançamentos tenham tido soma $7$ durante $24$ horas?
\end{enumerate}
\end{exercise}

\begin{exercise}
Imagine um modelo idealizado com $M$ eleitores, dos quais $M_A$
pretendem votar no candidato $A$. Suponha que seja possível sortear um
desses eleitores ao acaso, e de forma equiprovável.
Definimos
$$
  X=
\begin{cases}
1, & \mbox{caso o eleitor sorteado vá votar no candidato $A$},
\\
0, & \mbox{caso contrário}.
\end{cases}
$$

Deseja-se estimar a proporção $p=\frac{M_A}M$ de eleitores do candidato $A$,
que é desconhecida.
Para isso, repete-se este processo $N$ vezes, obtendo-se $X_1,\dots,X_N$.
Para estimar o valor de $p$ considera-se
$$
  \widehat p_N = \frac {X_1+\cdots+X_N}N.
$$
Supomos \textsl{a priori} que $p$ é bem próximo de $\frac{1}{2}$, de forma que
$VX \approx \frac{1}{4}$.
Se entrevistamos $N=2500$ eleitores, calcule aproximadamente a probabilidade de
essa pesquisa cometer um erro $|\widehat p_N - p|$ maior que $0,01$.
\end{exercise}

\begin{exercise}
Se lançamos 10.000 vezes uma moeda honesta, calcule aproximadamente a
probabilidade de que o número de vezes que se obtém coroa seja no mínimo 4.893
e no máximo 4.967.
\end{exercise}

\begin{exercise}
\end{exercise}


\clearpage
\section{Estimadores}

\begin{exercise}
Considere a variável aleatória discreta X, com distribuição uniforme sobre os inteiros $1, 2, 3,..., \theta$, isto é, 
\begin{center}
    \begin{tabular}{c|c|c|c|c|c}
    \hline
        X &1&2&3&...&$\theta$  \\
        \hline
         $p(x)$&$\frac{1}{\theta}$&$\frac{1}{\theta}$&$\frac{1}{\theta}$&$...$&$\frac{1}{\theta}$\\
    \hline
    \end{tabular}
\end{center}
Aqui, $\theta$ é um número inteiro positivo (por exemplo, $\theta = 10$). Uma amostra casual simples $X_1, X_2, X_3,..., X_n$ é selecionada e considera-se o seguinte estimador de $\theta$: $$\hat{\theta}= 2\overline{X}-1 \text{ onde 
 }\overline{X} = \frac{1}{n} \sum_{i=1}^{n}X_{i} $$
\begin{enumerate}
\item
Calcule $E(\hat{\theta})$ e $Var(\hat{\theta})$
\begin{solution}
$$
E(\hat{\theta}) = 2E(\overline{X}) - 1 = 2E\left(\frac{\sum_{i=1}^{n}{X_i}}{n}\right) - 1 = 2\frac{\sum_{i=1}^{n}E(X_i)}{n} - 1 = \frac{2n\left(\frac{\theta+1}{2}\right)}{n}-1 = \theta$$
$$
Var(\hat{\theta}) = Var(2\overline{X}-1) = Var(2\overline{X}) = 4\, Var\left(\frac{\sum_{i=1}^{n}{X_i}}{n}\right) = \frac{4}{n^{2}}\left(\sum_{i=1}^{n}Var(X_i)\right)
$$
$$
= \frac{4}{n} \cdot \frac{\theta^{2}-1}{12} = \frac{\theta^{2}-1}{3n}
$$
\end{solution}
\item 
$\hat{\theta}$ é não-enviesado? É consistente? Por quê?
\begin{answer}
$\hat{\theta}$ é não-enviesado e consistente, pois $E(\hat{\theta}) = \theta$ e $\lim_{n\to\infty} Var(\hat{\theta}) = 0$
\end{answer}
\item 
Se $n = 6$ e a amostra analisada for $x_1 = x_2 = x_3 = x_4 = x_5 = 1$ e $x_6 = 2$, qual é a estimativa de $\theta$?
Essa estimativa foi razoável?
\begin{answer}
Calculamos
$\hat{\theta} = \frac{4}{3}$.
Essa estimativa não é razoável, pois a observação $x_6=2$ sozinha já nos diz que $\theta \geq 2$ com certeza.
\end{answer}
\end{enumerate}
\end{exercise}

\begin{exercise}
\end{exercise}

\clearpage
\section{Distribuição $t$}

\begin{exercise}
Com o auxílio de uma tabela $t$-Student e uma calculadora, calcule fazendo interpolação linear:
\begin{enumerate}
\item $P(-1,87 \leq T_{5} \leq 1,87)$.
\begin{solution}
Como 1,87 não está na tabela, podemos utilizar os valores vizinhos  $P(-1,476 < x < 1,476 ) = 0,8$ e $P(-2,015 < x < 2,015) = 0,9$ para obter via interpolação linear que
\[
P(-1,87 < x < 1,87)
=
0,8 + \frac{1,87-1,476}{2,015-1,476} \cdot (0,9-0,8)
=
0,873
.
\]
\end{solution}
\item $P( T_{9} < 0,6 )$.
\begin{answer}
0,72
(interpolando da tabela)
\end{answer}
\item $P(-1,5 < T_{14} < 2,0)$.
\begin{solution}
Pela simetria da distribuição $T$, podemos consultar a tabela usando a propriedade de que
\[
P(T>t) = P(T<-t) = \tfrac{1}{2} \, P( |T| > t)
\]
para $t>0$.
Assim, obtemos diretamente da tabela
$P(T_{14}<-1,345)=0,10$ e $P(T_{14}<-1,761)=0,05$.

Fazendo interpolação linear, obtemos
$P(T_{14}<-1,5) = 0,0813$.

De forma semelhante,
$P(T_{14}>1,761)=0,05$ e $P(T_{14}>2,145)=0,025$.
Novamente, fazendo interpolação linear, obtemos
$P(T_{14}>2,0) = 0,034$.

Assim,
$P(T_{14}<2,0) = 1 - P(T_{14}>2,0) = 0,965$.

Finalmente,
$P(-1,5 < T_{14} < 2,0) = P(T_{14}<2,0) - P(T_{14}<-1,5) = 0,88$
\end{solution}
\item O valor de $a$ tal que $P(T_{9} > a) = 0,03$.
\begin{solution}
2,176
\end{solution}
\item O valor de $c$ tal que $P(|T_{11}| \leq c) = 0,92$
\begin{answer}
1,958
\end{answer}
\end{enumerate}
\end{exercise}

\begin{exercise}
\end{exercise}

\clearpage
\section{Distribuição $\chi^2$}

\begin{exercise}
Utilizando a tabela Qui-Quadrado e uma calculadora, determine:
\begin{enumerate}
    \item $P(\chi_{7}^{2} \geq 13)$
\begin{answer}
0,074
(interpolando da tabela)
\end{answer}
    \item $P( \chi_{23}^{2}\leq 29)$
\begin{answer}
0,8
(interpolando da tabela)
\end{answer}
    \item $P(\chi_{12}^{2} \leq 10)$
\begin{answer}
0,384
(interpolando da tabela)
\end{answer}
    \item $P(12 \leq \chi_{17}^{2} \leq 30,2)$
\begin{answer}
0,775
(não precisa interpolar)
\end{answer}
    \item O valor de $a$ tal que $P(\chi_{13}^{2} \geq a) = 0,06$
\begin{solution}
Como 0,06 não está na primeira linha da tabela $\chi^2$, tomamos os vizinhos 0,05 e 0,10.
Observamos que $P(\chi_{13}^{2} \geq 22,36) = 0,05$ e $P(\chi_{13}^{2} \geq 19,81) = 0,1$, e usamos interpolação linear para obter
\[
a
=
22,36+\frac{0,06-0,05}{0,10-0,05} \cdot (19,81-22,36)
=
21,85
.
\]
\end{solution}
    \item O valor de $b$ tal que $P(\chi_{4}^{2} > b) = 0,03$
\begin{answer}
10,77
(interpolando da tabela)
\end{answer}
\end{enumerate}
\end{exercise}

\begin{exercise}
\end{exercise}


\end{document}
